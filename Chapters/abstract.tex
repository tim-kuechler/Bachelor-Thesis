\noindent\textbf{Semantische Bildsynthese mit Score-basierten, generativen Modellen:}\\[0.3cm]
\noindent Die künstliche Bilderzeugung ist ein zukunftsweisendes Forschungsgebiet der modernen Informatik. Wenn der Datensatz, auf dessen Basis ein Modell Bilder generieren soll, sehr kompliziert ist, oder die semantische Bildinformation des generierten Bildes beeinflusst werden soll, wird semantische Bildsynthese benötigt. In dieser Arbeit zeigen wir einen Ansatz, der das neue und vielversprechende Konzept der Score-basierten generativen Modelle für die semantische Bildsynthese nutzt. Wir vergleichen unsere Ergebnisse auf dem Cityscapes-Datensatz \cite{cityscapes} quantitativ mit State-of-the-Art-Modellen wie CRN \cite{crn}, pix2pixHD \cite{pix2pixHD} und SPADE \cite{spade} unter Verwendung von FID-Scores sowie der Pixelgenauigkeit und des mittleren IoU. Darüber hinaus generieren wir hochauflösende Landschaftsbilder mit Bildern aus Flickr. Abschließend zeigen wir die Grenzen unseres Ansatzes am ADE20K-Datensatz \cite{ade20k} und weisen auf Herausforderungen hin, deren Lösung in der zukünftigen Forschung die semantische Bildsynthese mit Score-basierten generativen Modellen zu einem leistungsfähigen, vielseitigen und einfach zu verwendenden Framework machen könnte.\\[0.3cm]
\noindent\textbf{Semantic Image Synthesis with Score-Based Generative Models:}\\[0.3cm]
\noindent Artificial image generation is a cutting-edge field of research in modern computer science. If the dataset from which a model is to generate images is very complicated or the semantic image information of the generated image is to be influenced, semantic image synthesis is needed. In this thesis, we show an approach using the new and promising concept of score-based generative models for semantic image synthesis. We quantitatively compare our results on the Cityscapes dataset \cite{cityscapes} with state-of-the-art models like CRN \cite{crn}, pix2pixHD \cite{pix2pixHD} and SPADE \cite{spade} using FID scores as well as pixel accuracy and mean IoU. Furthermore, we generate high resolution landscape images using images scraped from Flickr. Finally, we show the limitations of our approach on the ADE20K dataset \cite{ade20k} and point out challenges whose solution in future research could make semantic image synthesis with score-based generative models a powerful, versatile and easy-to-use framework.